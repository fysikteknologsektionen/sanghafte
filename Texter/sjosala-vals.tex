\section{Sjösala vals}

Rönnerdahl han skuttar med ett skratt ur sin säng.\\
Solen står på Orrberget. Sunnanvind brusar.\\
Röhnnerdahl han valsar över Sjösala äng.\\
Hör min vackra visa, kom, sjung min refräng!\\
Tärnan har fått ungar och dyker i min vik,\\
ur alla gröna dungar hörs finkarnas musik,\\
och se så många blommor\\
som redan slagit ur på ängen.\\
Gullviva, mandelblom, kattfot och blå viol.\\

Rönnerdalh han virvlar sina lurviga ben\\
under vita skjortan som viftar kring vadorna.\\
Lycklig som en lärka uti majsolens sken,\\
sjunger han för ekorrn, som gungar på gren!\\
Kurre, kurre, kurre! Nu dansar Rönnerdahl!\\
Kokó! Och göken ropar uti hans gröna dal\\
och se så många blommor\\
som redan slagit ur på ängen.\\
Gullviva, mandelblom, kattfot och blå viol.\\

Rönnerdahl han binder utav blommor en krans,\\
binder den kring året, det gråa och rufsiga,\\
valsar in i stugan och har lutan till hands,\\
väcker frun och barnen med drill och kadans.\\
Titta, ropar ungarna, Pappa är en brud\\
med blomsterkrans i håret\\
och nattskjorta till skrud!\\
Och se så många blommor\\
som redan slagit ur på ängen.\\
Gullviva, mandelblom, kattfot och blå viol.\\

Rönnerdahl är gammal, men han valsar ändå!\\
Rönnerdahl har sorger och ont om sekiner.\\
Sällan får han rasta, han får slita för två.\\
Hur han klara skivan kan ingen förstå -\\
ingen, utom tärnar i viken (hon som dök)\\
och ekorren och finken och vårens första gök\\
och blommorna, de blommor\\
som redan slagit ut på ängen.\\
Gullviva, mandelblom, kattfot och blå viol.\\