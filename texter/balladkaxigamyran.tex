\begin{song}{Balladen om den kaxiga myran}{balladkaxigamyran}
\begin{vers}
Jag uppstämma vill min lyra,\\
fast det blott är en gitarr,\\
och berätta om en myra,\\
som gick ut att leta barr.\\
Han gick ut i morgondiset,\\
sen han druckit sin choklad\\
och försvann i lingonriset\\
//:både mätt och nöjd och glad://\\
\end{vers}
\begin{vers}
Det var långan väg att vandra\\
det var långt till närmsta tall.\\
Han kom bort ifrån dom andra\\
men var glad i alla fall.\\
Femti meter ifrån stacken\\
just när solnedgången kom,\\
hitta' han ett barr på marken\\
//:som han tyckte mycket om://\\
\end{vers}
\begin{vers}
För att lyfta fick han stånka,\\
han fick spänna varje lem,\\
men så började han kånka\\
på det fina barret hem.\\
När han gått i fyra timmar\\
kom han till en ölbutelj,\\
han såg allting som i dimma\\
//:bröstet hävdes som en bälg://\\
\end{vers}
\begin{vers}
Den låg kvar sen förra lördan.\\
- Jag skall släcka törsten min,\\
tänkte han och lade bördan\\
utanför och klättra' in.\\
Han drack upp den sista droppen\\
som fanns kvar i den butelj.\\
Och sedan slog han sig för kroppen\\
//:och skrek ut: - Jag är en älg!://\\
\end{vers}
\begin{vers}
- Ej ett barr jag drar till tjället,\\
nu så ska jag tamejfan\\
lämna skogen och i stället\\
vända upp och ner på stan.\\
Men han kom aldrig till staden,\\
något spärrade han stig,\\
en koloss där låg bland bladen\\
//:och vår myra hejdar sig://\\
\end{vers}
\begin{vers}
Den var hiskelig att skåda,\\
den var stor och den var grå,\\
och vår myra skrek: - Anåda,\\
om du hindrar mig att gå!\\
Han for ilsken på kolossen\\
som låg utsträckt i hans väg.\\
Men vår myra kom ej loss sen,\\
//:han satt fast som i en deg://\\
\end{vers}

\newp

\begin{vers}
Sorgligt slutar denna sången.\\
Myran stretade och drog,\\
men kolossen höll'en fången\\
tills han svalt ihjäl och dog.\\
Undvik alkoholens yra:\\
Du blir stursk, men kroppen loj,\\
och om Du är född till en myra\\
//:- brottas aldrig med ett TOY://\\
\end{vers}
\end{song}
